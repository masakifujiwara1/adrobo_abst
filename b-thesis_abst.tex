\documentclass[10pt]{jarticle}
\usepackage{float}
\usepackage{adrobo_abst}
\usepackage[dvipdfmx]{graphicx}
\usepackage{amssymb,amsmath}
\usepackage{bm}
\usepackage[superscript]{cite}
\usepackage{enumerate}
\usepackage{url}
%\usepackage[absolute]{textpos}

\renewcommand\citeform[1]{(#1)}

\begin{document}
    
    \makeatletter
    \doctype{2022年度卒業論文概要}
    \title{視覚と行動のend-to-end学習により経路追従行動を\\オンラインで模倣する手法の提案}{(目標方向による経路選択機能の追加と検証)}
    \etitle{A proposal for an online imitation method of path-tracking
    behavior by end-to-end learning of vision and action}{(Addition and verification of path selection function by target direction)}
    
    \author{19C1101\hspace{.5zw}藤原柾}
    \eauthor{Masaki FUJIWARA}
    
    \makeatother
    
    \abstract{When preparing the manuscript, read and observe carefully this sample as well as the instruction manual for the manuscript of the Transaction of Japan Society of Mechanical Engineers. This sample was prepared using MS-word. Character size of the English title is 14 pts of Times New Roman as well as sub-title. The name is 12 pts. The address of the first author and the abstract is 10 pts of Times New Roman. Character spacing of the abstract is narrowed by 0.2 pts preferably.}
    
    \keywords{End-to-end learning, Navigation, Target direction}
    
    \maketitle
    
    \supervisor{指導教員:林原靖男 教授}
    
    \section{緒\hspace{2zw}言}

    近年, 機械学習を用いた自律走行に関する研究がされている. Bojarski\cite{bojarski}らは, 人間のドライバーが操作するステアリングの角度と前方カメラ画像を用いて模倣学習を行った. 加えて, 訓練したネットワークに画像を入力し, 生成される操舵指令を用いて走行を行う手法を提案した. また, 岡田ら\cite{okada}はLiDAR, オドメトリを入力とした地図を用いたルールベース制御器による経路追従行動を, 前方カメラ画像を用いてend-to-endで模倣学習した. その結果, カメラ画像に基づいてロボットが学習した経路を周回可能であることが確認されている. \\本研究では, 岡田ら\cite{okada}の研究(以下, 「従来手法」と称する)をベースとして, 分岐路において「直進」, 「左折」などのコマンドによる制御で, 任意の経路を選択可能にする機能の追加を提案する. また, シミュレータ上での実験を実環境に移す際に, 問題となった学習時間の長さについて, 2 つのアプローチの提案と検証を行う. さらに, 実環境における提案手法の有効性を検証することを目的とする.
    % --------------------------------------------------------------------------------------------------------------------------------------------------------------------
    % \section{記号・単位の書き方}%===========================
    % \begin{table}[!h]
    %     \begin{tabular}{lcl}
    %         $L$ & : & 長さ [m] \\
    %         $t$ & : & 時間 [s] \\
    %         $x$ & : & 流れ方向の座標 [m] \\
    %         $\alpha$ & : & 熱伝達率 [$\mathrm{W/(m^2\cdot K)}$] \\
    %         $Re$ & : & レイノルズ数 \\
    %         $\bm{R}$ & : & 回転行列 \\
    %         $\bm{t}$ & : & 並進ベクトル \\
    %     \end{tabular}
    % \end{table}
    
    
    
    \section{従来手法}%===========================

    岡田らの従来手法に関して紹介する. 
    \reffig{sample-fig}に, 経路追従行動を視覚に基づいてオンラインで模倣するシステムを示す. 
    手法は機械学習により, 学習器の訓練を行う「学習フェーズ」と訓練した結果を検証する「テストフェーズ」に分かれる. 

    \begin{center}
        \begin{figure}[!b]
            \includegraphics[width=0.45\textwidth]{./fig/convsys.png}
            \caption{Sample of clear figure}
            \label{fig:sample-fig}
        \end{figure}
    \end{center}

    % \newpage
    
    \subsection{学習フェーズ}
    
    学習フェーズは, 模倣学習によって学習器の訓練を行うフェーズである. LiDARとオドメトリを入力とする地図を用いたルールベース制御器で自律走行する. この経路追従行動を, カメラ画像を用いたend-to-endで模倣学習する.

    \subsection{テストフェーズ}

    テストフェーズは, 訓練後の学習結果を評価するフェーズである. 学習器にカメラ画像を入力し, 出力されるヨー方向の角速度を用いて自律走行する.
    % なお, \reffig{sample-fig}の提案手法と記載した部分は, 次章で述べるように, 本研究で追加した機能となる.

    \section{提案手法}

    経路選択機能の追加を目的として, データセットと学習器の入力へ「直進」, 「左折」などの目標方向を追加する. なお, 追加した要素以外は従来手法と同様である. \reffig{propose_sys}に, 提案手法のシステムを示す. 学習フェーズでは, 目標方向の生成機能を追加した地図ベースの制御器を用いる. テストフェーズでは, 学習器の出力を用いた走行において, 目標方向によって任意の経路を選択する.
    

    % なお, \reffig{propose_sys}では, 従来手法のシステムに追加した部分を提案手法と記載している. 

    \begin{center}
        \begin{figure}[!b]
            \includegraphics[width=0.45\textwidth]{./fig/system2.png}
            \caption{Sample of clear figure}
            \label{fig:propose_sys}
        \end{figure}
    \end{center}

    本研究で用いた目標方向と学習器へ入力するデータ形式を, \reftab{target_direction}に示す. 目標方向を分岐路において「直進(Go straight)」, 「左折(Turn left)」, 「右折(Turn right)」の3コマンドを用いる.

     \begin{table}[t]
        \caption{Sample of expression of values}
        \label{tab:target_direction}
        \begin{center}
            \vskip 1zh
            \begin{tabular}{|c|c|}
                \hline
                Target direction & Data \\ \hline
                Go straight & [100, 0, 0] \\ \hline
                Turn left & [0, 100, 0] \\ \hline
                Trun right & [0, 0, 100] \\ \hline
            \end{tabular}
        \end{center}
    \end{table}

    \section{実験}
    
    
    
    % \section{図及び写真・表の作成に関して}%===========================
    % \begin{enumerate}
    %     \setlength{\parskip}{0cm} % 段落間
    %     \setlength{\itemsep}{0cm} % 項目間
    %     \item 本文中では,\reffig{sample-fig},\reftab{sample-tab}のように日本語で書く.写真は,図として扱う.
    %     \item 番号・説明(キャプション)などは,図・写真についてはその下に,表についてはその上に書く.
    %     \item 本文と,図・表の間は1行以上の空白を空けて,見やすくする.
    %     \item 図中・表中の説明及びキャプションはすべて英語で書く(最初の文字は大文字とする).
    %     \item 図及び表がl列(片側)に収まらない場合2列(両側)にまたがって書くことができる. 
    %     \item 図及び表の横に空白ができても,その空白部には本文を記入してはならない.
    % \end{enumerate}
    
    % \begin{table}[t]
    %     \caption{Sample of expression of values}
    %     \label{tab:sample-tab}
    %     \begin{center}
    %         \vskip 1zh
    %         \begin{tabular}{|c|c|}
    %             \hline
    %             Recommend & Not recommend \\ \hline
    %             $0.357$ & $.357$ \\ \hline
    %             $3.141\ 6$ & $3.141,6$ \\ \hline
    %             $3.141\ 6 \times 2.5$ & $3.141\ 6 \cdot 2.5$ \\ \hline
    %         \end{tabular}
    %     \end{center}
    % \end{table}
    
    % \section{数式の書き方}%===========================
    
    % 式番号は,式と同じ行に右寄せして( )の中に書く.また,本文で式を引用するときは,\refeqn{sample-eq1}のように書く.
    % \begin{equation}
    %     \gamma(t) = \frac{ji}{N} \label{eq:sample-eq1}
    % \end{equation}
    % \begin{equation}
    % \bar{C}(t)=\frac{1}{N}\sum_{i=1}^{N}C_i(t) \label{eq:sample-eq2}
    % \end{equation}
    
    %\begin{table}[!b] \notag
    %\begin{minipage}{\textwidth}
    %\begin{tabular*}{\textwidth}{@{\extracolsep{\fill}}lr}
    %{\footnotesize
    %$\displaystyle 
    %u^*_{i,j} = u^n_{i,j} - \Delta t \left\{u^n_{i,j}\frac{u^n_{i+1,j} - u^n_{i-1,j}}{2\Delta x} + v^n_{i,j}\frac{u^n_{i,j+1} - u^n_{i,j-1}}{2\Delta y} + \frac{1}{Re}\left(\frac{u^n_{i+1,j} - 2u^n_{i,j} + u^n_{i-1,j}}{(\Delta x)^2} + \frac{u^n_{i,j+1} - 2u^n_{i,j} + u^n_{i,j-1}}{(\Delta y)^2}\right) \right\}
    %$}
    %& $\inlineTag\label{eq:long-eq} $
    %\end{tabular*}
    %\end{minipage}
    %\end{table}
    
    % 式を書くときは,2文字分空白を空ける.
    % また,必要行数分を必ず使うようにして書く.
    % 3行必要とする式を2行につめて書いたり,2行に分かれる式を1行に収めたりしない.
    % なお,本文と式,式相互間は1行以上の空白を空けて,見やすくする.
    % ポイント数は本文に準じるものとするが,添え字等が小さく読みにくくなるときは適宜拡大する.
    
    %式はなるべく片側に書くことが望ましいが,両側にまたがる場合は,読む順序に混乱を生じないように,そのページの式の上,または下の本文全部を両方にまたがるように書かなければならない.
    %本見本では\refeqn{long-eq}のようにページの最上段もしくは最下段に配置している場合は,上記のような混乱は生じ得ないので以下の文章は2段組で続けることができる.
    %ただし,所望の位置に表示されない,文字が重なるなどレイアウト上の多くの問題が生じるため,極めて推奨しない.
    
    % \section{引用文献の書き方}%===========================
    % 本文中の引用箇所には,右肩に小括弧をつけて,通し番号を付ける.例えば,文献\cite{工大2005}や,文献\cite{Shibutani2004, Handbook1979, Kikuchi2017, Adrobo2019}のようにする.
    
    % 引用文献は,英文で記述されているもの(文献\cite{Shibutani2004}など)は英文で書き,本文末尾に引用順にまとめて書く.専門的な書籍(文献\cite{Handbook1979}など)についても引用しても良い.
    % Web上の資料を引用する場合,例えばオンラインジャーナルなどの場合は文献\cite{Kikuchi2017}のように,webページの場合は文献\cite{Adrobo2019}のように,それぞれ参考文献として記載して引用する.この時,URLとともに参照日を記載すること.ただし,webページの場合は個人の技術ブログなどのように第3者による十分な審査が行われていないものの引用は行ってはいけない.公的な機関が発行しているページであっても,その永続性の問題から必要最小限に留めることを推奨する.
        
    \section{結\hspace{2zw}言}%===========================
    
    本研究では, 経路追従行動をカメラ画像を用いた end-to-end 学習で模倣する岡田らの従来
    手法をベースに, データセットと学習器の入力へ目標方向を加えることで, 経路選択をする機
    能の追加を提案した. また, シミュレータ上での実験を実環境に移す際に, 問題となった学習時
    間の長さについて, 2 つのアプローチを試みることで学習時間を大幅に削減した. 加えて, 実環
    境での実験を行い, 有効性の検証を行った. 実験結果より, 学習器へ目標方向を与えることで,
    指定した経路へ走行する挙動が確認できた.
    
    \vspace{5truemm}
    {\footnotesize
        \begin{thebibliography}{99}
            
            \bibitem{bojarski}
            Mariusz Bojarski et al: “End to End Learning for Self-
            Driving Cars”, 
            arXiv: 1604.07316,(2016)
            
            \bibitem{okada}
            岡田眞也, 清岡優祐, 上田隆一, 林原靖男: “視覚と行動のend-to-end 学習により経路追従行動 をオンラインで模倣する手法の提案”, 
            計測自動制御学会 si 部門講演会 sice-si2020予稿集,pp.1147–1152(2020).

        \end{thebibliography}
    }
    \normalsize
    
\end{document}
